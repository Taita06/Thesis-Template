Các kết quả của mô hình OFDMA-IDMA đề xuất được mô phỏng và đánh giá để đánh giá hiệu suất so với các kỹ thuật hiện có trong 802.11ax. Các tham số mô hình OFDMA-IDMA chi tiết được nêu trong Bảng \ref{tab:Para}. Bội số SF sẽ giảm số bit dữ liệu để điều chỉnh kích thước gói của hệ thống OFDMA-IDMA theo tiêu chuẩn 802.11.

\begin{table}[H]
	\normalsize
	\centering
	\caption{Thông số mô phỏng}
	\label{tab:Para}
	\begin{tabular}{|l|l|}
		\hline
		\textbf{Parameters}   & \textbf{Value}       \\ \hline
		Mô hình kênh          & TGax Channel Model B \\ \hline
		Băng thông            & 20 MHz               \\ \hline
		Số lượng RU chơ IDMA  & 4                    \\ \hline
		Số lượng STA          & 8, 12, 16, 20, 24    \\ \hline
		Hệ số trải (SF)       & 2                    \\ \hline
		Kiểu điều chế         & 16-QAM               \\ \hline
		Số lượng ăng-ten thu  & 4, 8                 \\ \hline
		Số vòng lặp           & 4                    \\ \hline
		Mã hóa kênh           & LDPC                 \\ \hline
		Packet length         & 1000 Bytes           \\ \hline
		Số lượng lặp mô phỏng & 1000                 \\ \hline
	\end{tabular}
\end{table}
%=====================
Hình \ref{fig:IT} cho thấy sự tác động của số lần lặp đến hiệu suất BER của OFDMA-IDMA. Khi số lần lặp tăng lên, BER được cải thiện rõ rệt. Tuy nhiên, điều cần thiết là phải xem xét sự đánh đổi với độ trễ, độ trễ có xu hướng tăng tuyến tính theo số lần lặp, để tạo sự cân bằng giữa độ trễ và chất lượng tín hiệu là rất quan trọng.
Đối với 802.11 \acrshort{OFDM} PHY, lý tưởng nhất là quá trình nhận phải được hoàn thành trong vòng 16 $\mu s$. Xem xét các yếu tố thực tế, bao gồm độ trễ xử lý đường truyền và \acrshort{MAC}, độ trễ xử lý mục tiêu nhận khoảng 10 $\mu s$ được khuyến nghị trong \cite{UL-OFDM-IDMA}.
Các tác giả này đã nghiên cứu mối quan hệ giữa số lần lặp và độ trễ trong các kịch bản OFDM-IDMA. Đối với SF bằng 2, độ trễ của hệ thống là khoảng 6 $\mu s$ cho 4 lần lặp, trong khi nó vượt quá 10 $\mu s$ cho 8 lần lặp. Do đó, báo cáo này đã chọn ra 4 lần lặp lại là một sự thỏa hiệp phù hợp để đáp ứng các yêu cầu về chất lượng và độ trễ.

\begin{figure}
	\pgfplotsset{width=0.75\linewidth,height=10cm,
		x label style={at={(axis description cs:0.5 , -0.05)},anchor=north},
		y label style={at={(axis description cs:-0.125, 0.5)},anchor=south},
		xmajorgrids,
		ymajorgrids,
	}
	\pgfplotsset{yticklabel style={text width=2em,align=right}}
	\centering
	% This file was created with tikzplotlib v0.10.1.
\begin{tikzpicture}

\definecolor{darkturquoise0191191}{RGB}{0,191,191}
\definecolor{darkviolet1910191}{RGB}{191,0,191}
\definecolor{green01270}{RGB}{0,127,0}
\definecolor{lightgray204}{RGB}{204,204,204}
\definecolor{darktangerine}{RGB}{1.0, 0.66, 0.07}

\begin{axis}[
legend cell align={left},
legend style={
  fill opacity=0.8,
  draw opacity=1,
  text opacity=1,
  at={(0.03,0.03)},
  anchor=south west,
  draw=lightgray204
},
log basis y={10},
tick pos=left,
xlabel={SNR [dB]},
xmajorgrids,
xmin=9.5, xmax=20.5,
xminorgrids,
xtick style={color=black},
ylabel={BER},
ymajorgrids,
ymin=2.16482046582444e-05, ymax=0.513035557901541,
yminorgrids,
ymode=log,
ytick style={color=black},
% ytick={1e-06,1e-05,0.0001,0.001,0.01,0.1,1,10},
% yticklabels={
%   \(\displaystyle {10^{-6}}\),
%   \(\displaystyle {10^{-5}}\),
%   \(\displaystyle {10^{-4}}\),
%   \(\displaystyle {10^{-3}}\),
%   \(\displaystyle {10^{-2}}\),
%   \(\displaystyle {10^{-1}}\),
%   \(\displaystyle {10^{0}}\),
%   \(\displaystyle {10^{1}}\)
% }
]
\addplot [semithick, blue, mark=o, mark size=3, mark options={solid}]
table {%
10 0.342706880040323
12 0.273475378024194
14 0.217120236895161
16 0.18451154233871
18 0.163001108870968
20 0.154058669354839
};
\addlegendentry{Số lần lặp = 1}
\addplot [semithick, darkviolet1910191, mark=triangle, mark size=3, mark options={solid}]
table {%
10 0.22259236391129
12 0.103527570564516
14 0.0351042842741935
16 0.0120323588709677
18 0.00708961693548387
20 0.00545539314516129
};
\addlegendentry{Số lần lặp = 2}
\addplot [semithick, darkturquoise0191191, mark=diamond, mark size=3, mark options={solid}]
table {%
10 0.196327394153226
12 0.0706490675403226
14 0.0148261340725806
16 0.00169695060483871
18 0.000602318548387097
20 0.000296421370967742
};
\addlegendentry{Số lần lặp = 4}
\addplot [semithick, green01270, mark=square, mark size=3, mark options={solid}]
table {%
10 0.194996824596774
12 0.0680136088709677
14 0.0130689516129032
16 0.00123087197580645
18 0.000225478830645161
20 0.000107258064516129
};
\addlegendentry{Số lần lặp = 8}
\addplot [ultra thick, loosely dashed, darktangerine, forget plot]
  table {%
  9.5 0.001
  20.5 0.001
  };
  \node[] at (axis cs: 12, 1.5e-3) {BER tham chiếu};
\end{axis}

\end{tikzpicture}

	\caption{Ảnh hưởng của số lần lặp với STA = 20, NRx = 8.}
	\label{fig:IT}
	% \vspace{-1em}
\end{figure}
%==========================

Hình \ref{fig:bersec4} mô tả BER trung bình cho các kỹ thuật 802.11ax sử dụng bộ cân bằng Lỗi bình phương trung bình tối thiểu (\acrshort{MMSE}) ở bộ thu với số lượng STA và ăng-ten thu (NRx) khác nhau trong băng thông 20 MHz.
Hình \ref{fig:BER MU-MIMO} thể hiện BER của kỹ thuật MU-MIMO có thể thấy được MU-MIMO rất nhạy cảm với nhiễu MAI. Ngay cả trong điều kiện tối ưu với 4 STA được truyền tới AP 8 ăng-ten, việc đạt được BER tham chiếu vẫn còn khó khăn.
Hình \ref{fig:BER OFDMA} minh họa BER của OFDMA, cho thấy ảnh hưởng MAI giảm khi STA truyền trong các RU khác nhau. Đối với 4 STA và NRx là 8, BER giảm xuống dưới $10^{-3}$ ở mức 20~[dB]. Tuy nhiên, khi số lượng STA tăng lên thì BER của OFDMA cũng tăng do nhiễu tích lũy tăng lên. Lý do là tỷ lệ giữa số điểm FFT với số lượng sóng mang trên mỗi RU sẽ tăng khi số lượng STA tăng, dẫn đến hiệu ứng tích lũy nhiễu sẽ lớn hơn khi số lượng STA tăng.
Hình \ref{fig:BER OFDMA-MU-MIMO} hiển thị BER trung bình của OFDMA-MU-MIMO với trường hợp số RU là 2, đây là kích thước RU tối thiểu để triển khai OFDMA-MU-MIMO ở 20 MHz trong 802.11ax \cite{IEEEStd}. Mặc dù OFDMA-MU-MIMO mang lại hiệu suất phổ cao hơn OFDMA khi số lượng STA tăng lên, nhưng nó phải đối mặt với những thách thức trong việc đạt được BER tham chiếu, nhưng BER của nó vẫn vượt trội hơn so với phương án sử dụng MU-MIMO.

Hình \ref{fig:BER OFDMA-IDMA} minh họa BER trung bình trong OFDMA-IDMA được đề xuất với 4 RU. Có thể thấy tốc độ dữ liệu của OFDMA-IDMA sẽ giảm đi một nửa so với các kỹ thuật khác nêu trên khi chọn SF là 2, tuy nhiên chất lượng và số lượng STA tăng lên đáng kể so với các kỹ thuật đó. Khi đặt cạnh trường hợp chia RU và nhiều STA trong một RU như được mô tả trong Hình \ref{fig:BER OFDMA-MU-MIMO}, kỹ thuật OFDMA-IDMA được đề xuất sẽ vượt trội hơn. Xem xét trường hợp tốt nhất của OFDMA-MU-MIMO với 4 STA và 8 ăng-ten thu (STA~=~4, NRx = 8), BER xấp xỉ $10^{-2}$ tại SNR~=~20 [dB]. Tuy nhiên, đối với ba kỹ thuật trong hình \subref{fig:BER MU-MIMO}, \subref{fig:BER OFDMA}, \subref{fig:BER OFDMA-MU-MIMO}, giả sử việc đánh đổi việc giảm tốc độ bằng một nửa tốc độ dữ liệu và nhân đôi số lượng STA là hợp lý, vì vậy trường hợp tốt nhất OFDMA-IDMA (STA = 8 , NRx~=~8) đạt được hầu như không có lỗi ở SNR~=~20 [dB]. Tuy nhiên, so với OFDMA thông thường, trong đó trường hợp tốt nhất (STA~=~4, NRx = 8) mang lại BER khoảng $3\mathord{\times}10^{-4}$ tại SNR~=~20 [dB] đạt được BER tham chiếu thì rõ ràng OFDMA-IDMA vẫn cho kết quả tốt hơn nhiều (STA~=~8, NRx = 8).
Ngoài ra, kỹ thuật đề xuất còn triển khai MU đường lên trong 4 RU, điều mà MU-MIMO không thể đạt được trong 802.11ax. Về mặt hiệu quả phổ tần, việc triển khai OFDMA-IDMA ở 4 RU sẽ giảm hai lần so với 2 RU như MU-MIMO, nhưng chất lượng truyền dẫn cũng như số lượng STA sẽ tăng lên đáng kể.
Sự đánh đổi này đáng được xem xét trong trường hợp OFDMA-MU-MIMO tốt nhất (STA = 4, NRx = 8); BER của kỹ thuật này vẫn chưa thể đạt tới ngưỡng tham chiếu ở mức 20 dB. Đối với OFDMA-IDMA, khi xem xét tăng số lượng STA lên 20, BER vẫn xấp xỉ $3\mathord{\times}10^{-4}$, nghĩa là mặc dù tốc độ dữ liệu hoặc hiệu suất phổ giảm đi một nửa, số lượng STA tăng gấp 5 lần so với OFDMA. Ngay cả khi xem xét trường hợp STA~=~24 và Nrx = 8, BER của OFDMA-IDMA vẫn tiếp cận mức tham chiếu, cho thấy số lượng STA tăng gấp sáu lần so với trường hợp tốt nhất của OFDMA-MU-MIMO.


\begin{figure}
	\centering
	\pgfplotsset{width=\linewidth,height=10cm,
		x label style={at={(axis description cs:0.5 , -0.1)},anchor=north},
		y label style={at={(axis description cs:-0.2, 0.5)},anchor=south},
		xmajorgrids,
		ymajorgrids,
	}
	\pgfplotsset{yticklabel style={text width=2em,align=right}}
	
	\begin{subfigure}[]{0.45\linewidth}
		% This file was created with tikzplotlib v0.10.1.
\begin{tikzpicture}[scale = 1]

\definecolor{darkturquoise0191191}{RGB}{0,191,191}
\definecolor{darkviolet1910191}{RGB}{191,0,191}
\definecolor{goldenrod1911910}{RGB}{191,191,0}
\definecolor{green01270}{RGB}{0,127,0}
\definecolor{lightgray204}{RGB}{204,204,204}
\definecolor{darktangerine}{RGB}{1.0, 0.66, 0.07}

\begin{axis}[
legend cell align={left},
legend columns=2,
legend style={
  fill opacity=0.8,
  draw opacity=1,
  text opacity=1,
  at={(0.01,0.15)},
  anchor=west,
  draw=lightgray204
},
log basis y={10},
tick pos=left,
xlabel={SNR [dB]},
xmajorgrids,
xmin=9.5, xmax=20.5,
xminorgrids,
xtick style={color=black},
ylabel={BER},
ymajorgrids,
ymin=9.28649163113646e-06, ymax=0.572645816836016,
yminorgrids,
ymode=log,
ytick style={color=black},
% ytick={1e-05,0.0001,0.001,0.01,0.1,1,10},
% yticklabels={
%   \(\displaystyle {10^{-5}}\),
%   \(\displaystyle {10^{-4}}\),
%   \(\displaystyle {10^{-3}}\),
%   \(\displaystyle {10^{-2}}\),
%   \(\displaystyle {10^{-1}}\),
%   \(\displaystyle {10^{0}}\),
%   \(\displaystyle {10^{1}}\)
% }
]
\addplot [semithick, dashed, blue, mark=o, mark size=3, mark options={solid}]
table {%
10 0.495759402929494
12 0.493089219215492
14 0.484774515888779
16 0.462454288728898
18 0.41964957795432
20 0.367778891509434
};
\addlegendentry{\scriptsize STA=4, NRx=4}
\addplot [semithick, blue, mark=o, mark size=3, mark options={solid}]
table {%
10 0.487141633565045
12 0.462870562313803
14 0.393067744538232
16 0.272607032025819
18 0.162863424776564
20 0.0931232311320755
};
\addlegendentry{\scriptsize STA=4, NRx=8}
% \addplot [semithick, dashed, darkviolet1910191, mark=x, mark size=3, mark options={solid}]
% table {%
% 10 0.496456926514399
% 12 0.493548336643496
% 14 0.486620456802383
% 16 0.470138257199603
% 18 0.44634957795432
% 20 0.412961146971202
% };
% \addlegendentry{\scriptsize STA=5, NRx=4}
% \addplot [semithick, darkviolet1910191, mark=x, mark size=3, mark options={solid}]
% table {%
% 10 0.48865367428004
% 12 0.469117154915591
% 14 0.409146027805362
% 16 0.310444935451837
% 18 0.206696623634558
% 20 0.13252323733863
% };
% \addlegendentry{\scriptsize STA=5, NRx=8}
\addplot [semithick, dashed, darkviolet1910191, mark=x, mark size=3, mark options={solid}]
table {%
10 0.497733594008606
12 0.496036060079444
14 0.492732187189672
16 0.485523191823899
18 0.472735807679576
20 0.457692506620324
};
\addlegendentry{\scriptsize STA=6, NRx=4}
\addplot [semithick, darkviolet1910191, mark=x, mark size=3, mark options={solid}]
table {%
10 0.493223518702416
12 0.484519157563721
14 0.454731980304535
16 0.391884082257531
18 0.314623965574313
20 0.235754613538563
};
\addlegendentry{\scriptsize STA=6, NRx=8}
% \addplot [semithick, dashed, green01270, mark=square, mark size=3, mark options={solid,rotate=180}]
% table {%
% 10 0.498135675273088
% 12 0.496128670733437
% 14 0.494529401333522
% 16 0.489265534118315
% 18 0.481894240317776
% 20 0.473643513264293
% };
% \addlegendentry{\scriptsize STA=7, NRx=4}
% \addplot [semithick, green01270, mark=square, mark size=3, mark options={solid,rotate=90}]
% table {%
% 10 0.494164757412399
% 12 0.48773221378919
% 14 0.466358118172791
% 16 0.425929475812172
% 18 0.371793587742942
% 20 0.309308678535963
% };
% \addlegendentry{\scriptsize STA=7, NRx=8}
\addplot [semithick, dashed, darkturquoise0191191, mark=triangle, mark size=3, mark options={solid}]
table {%
10 0.498238021350546
12 0.49737166397716
14 0.496360430114201
16 0.494041506330685
18 0.488708974677259
20 0.484506501365442
};
\addlegendentry{\scriptsize STA=8, NRx=4}
\addplot [semithick, darkturquoise0191191, mark=triangle, mark size=3, mark options={solid}]
table {%
10 0.495683915714995
12 0.491379499751738
14 0.481505678997021
16 0.458321577085402
18 0.419660625620655
20 0.37813781653426
};
\addlegendentry{\scriptsize STA=8, NRx=8}
\addplot [ultra thick, loosely dashed, darktangerine, forget plot]
table {%
9.5 0.001
20.5 0.001
};
\node[] at (axis cs: 13,1.5e-3) {BER tham chiếu};
\end{axis}

\end{tikzpicture}

		\vspace{-1.5em}
		\caption{BER của MU-MIMO.}\label{fig:BER MU-MIMO}
	\end{subfigure}
	\begin{subfigure}[]{0.45\linewidth}
		% This file was created with tikzplotlib v0.10.1.
\begin{tikzpicture}[scale = 1]

\definecolor{darkturquoise0191191}{RGB}{0,191,191}
\definecolor{darkviolet1910191}{RGB}{191,0,191}
\definecolor{green01270}{RGB}{0,127,0}
\definecolor{lightgray204}{RGB}{204,204,204}
\definecolor{darktangerine}{RGB}{1.0, 0.66, 0.07}

\begin{axis}[
legend cell align={left},
legend columns=2,
legend style={
  fill opacity=0.8,
  draw opacity=1,
  text opacity=1,
  at={(0.01,0.15)},
  anchor=west,
  draw=lightgray204
},
log basis y={10},
tick pos=left,
xlabel={SNR [dB]},
xmajorgrids,
xmin=9.5, xmax=20.5,
xminorgrids,
xtick style={color=black},
ylabel={BER},
ymajorgrids,
ymin=9.28649163113646e-06, ymax=0.572645816836016,
yminorgrids,
ymode=log,
ytick style={color=black},
% ytick={1e-05,0.0001,0.001,0.01,0.1,1,10},
% yticklabels={
%   \(\displaystyle {10^{-5}}\),
%   \(\displaystyle {10^{-4}}\),
%   \(\displaystyle {10^{-3}}\),
%   \(\displaystyle {10^{-2}}\),
%   \(\displaystyle {10^{-1}}\),
%   \(\displaystyle {10^{0}}\),
%   \(\displaystyle {10^{1}}\)
% }
]
\addplot [semithick, dashed, blue, mark=o, mark size=3, mark options={solid}]
table {%
10 0.38003634375
12 0.24171484375
14 0.1196720625
16 0.05172925
18 0.019391
20 0.00756159375
};
\addlegendentry{\scriptsize STA=4, NRx=4}
\addplot [semithick, blue, mark=o, mark size=3, mark options={solid}]
table {%
10 0.2608909375
12 0.1283121875
14 0.042170625
16 0.009151875
18 0.0018371875
20 0.0003165625
};
\addlegendentry{\scriptsize STA=4, NRx=8}
\addplot [semithick, dashed, darkviolet1910191, mark=x, mark size=3, mark options={solid}]
table {%
10 0.3653874185502
12 0.244531814935065
14 0.137767682239635
16 0.0694292653752498
18 0.0330769410433317
20 0.01451406685502
};
\addlegendentry{\scriptsize STA=8, NRx=4}
\addplot [semithick, darkviolet1910191, mark=x, mark size=3, mark options={solid}]
table {%
10 0.273217060876623
12 0.145498864541708
14 0.0602944973464036
16 0.0212365223682567
18 0.00610429470529471
20 0.00145596714223277
};
\addlegendentry{\scriptsize STA=8, NRx=8}
\addplot [semithick, dashed, darkturquoise0191191, mark=triangle, mark size=3, mark options={solid}]
table {%
10 0.361804513888889
12 0.241939041666667
14 0.139245861111111
16 0.0715146805555556
18 0.0355674583333333
20 0.0165399444444444
};
\addlegendentry{\scriptsize STA=9, NRx=4}
\addplot [semithick, darkturquoise0191191, mark=triangle, mark size=3, mark options={solid}]
table {%
10 0.277957
12 0.147020666666667
14 0.0593184722222222
16 0.0205094305555556
18 0.00571811111111111
20 0.00159363888888889
};
\addlegendentry{\scriptsize STA=9, NRx=8}
\addplot [ultra thick, loosely dashed, darktangerine, forget plot]
  table {%
  9.5 0.001
  20.5 0.001
  };
\node[] at (axis cs: 13, 1.5e-3) {BER tham chiếu};
\end{axis}

\end{tikzpicture}

		\vspace{-1.5em}
		\caption{BER của OFDMA.}\label{fig:BER OFDMA}
	\end{subfigure}
	
	\vspace{1cm}
	%================================
	\begin{subfigure}[]{0.45\linewidth}
		% This file was created with tikzplotlib v0.10.1.
\begin{tikzpicture}[scale = 1]

\definecolor{darkturquoise0191191}{RGB}{0,191,191}
\definecolor{darkviolet1910191}{RGB}{191,0,191}
\definecolor{goldenrod1911910}{RGB}{191,191,0}
\definecolor{green01270}{RGB}{0,127,0}
\definecolor{lightgray204}{RGB}{204,204,204}
\definecolor{darktangerine}{RGB}{1.0, 0.66, 0.07}

\begin{axis}[
legend cell align={left},
legend columns=2,
legend style={
  fill opacity=0.8,
  draw opacity=1,
  text opacity=1,
  at={(0.01,0.15)},
  anchor=west,
  draw=lightgray204
},
log basis y={10},
tick pos=left,
tick align=inside,
xlabel={SNR [dB]},
xmajorgrids,
xmin=9.5, xmax=20.5,
xminorgrids,
xtick style={color=black},
ylabel={BER},
ymajorgrids,
ymin=9.28649163113646e-06, ymax=0.572645816836016,
yminorgrids,
ymode=log,
ytick style={color=black},
% ytick={1e-08,1e-07,1e-06,1e-05,0.0001,0.001,0.01,0.1,1,10,100}
% yticklabels={
%   \(\displaystyle {10^{-9}}\),
%   \(\displaystyle {10^{-8}}\),
%   \(\displaystyle {10^{-7}}\),
%   \(\displaystyle {10^{-6}}\),
%   \(\displaystyle {10^{-5}}\),
%   \(\displaystyle {10^{-4}}\),
%   \(\displaystyle {10^{-3}}\),
%   \(\displaystyle {10^{-2}}\),
%   \(\displaystyle {10^{-1}}\),
%   \(\displaystyle {10^{0}}\),
%   \(\displaystyle {10^{1}}\)
% }
]
\addplot [semithick, dashed, blue, mark=o, mark size=3, mark options={solid}]
  table {%
  10 0.475414311849512
  12 0.427380552788845
  14 0.328055577689243
  16 0.216280490537849
  18 0.126077981822709
  20 0.0715704961404383
  };
  \addlegendentry{\scriptsize STA=4, NRx=4}
  \addplot [semithick, blue, mark=o, mark size=3, mark options={solid}]
  table {%
  10 0.420903230826693
  12 0.281042517430279
  14 0.152157899651394
  16 0.0647917081673307
  18 0.0242504980079681
  20 0.0092628859561753
  };
  \addlegendentry{\scriptsize STA=4, NRx=8}
  \addplot [semithick, dashed, darkviolet1910191, mark=x, mark size=3, mark options={solid}]
  table {%
  10 0.485489402805445
  12 0.458627321962151
  14 0.398791131308101
  16 0.314991239209827
  18 0.237854262118194
  20 0.174035113711819
  };
  \addlegendentry{\scriptsize STA=6, NRx=4}
  \addplot [semithick, darkviolet1910191, mark=x, mark size=3, mark options={solid}]
  table {%
  10 0.43927214060425
  12 0.327331382802125
  14 0.202177851095618
  16 0.114883507636122
  18 0.0579899983399734
  20 0.0287592338977424
  };
  \addlegendentry{\scriptsize STA=6, NRx=8}
  \addplot [semithick, dashed, darkturquoise0191191, mark=triangle, mark size=3, mark options={solid}]
  table {%
  10 0.493769644858068
  12 0.485925491782868
  14 0.465015418015438
  16 0.427567000747012
  18 0.380129934947709
  20 0.33273046875
  };
  \addlegendentry{\scriptsize STA=8, NRx=4}
  \addplot [semithick, darkturquoise0191191, mark=triangle, mark size=3, mark options={solid}]
  table {%
  10 0.478601375747012
  12 0.430529927166335
  14 0.353783771165339
  16 0.247454494521912
  18 0.167064414218127
  20 0.103025740786853
  };
  \addlegendentry{\scriptsize STA=8, NRx=8}
\addplot [ultra thick, loosely dashed, darktangerine, forget plot]
  table {%
  9.5 0.001
  20.5 0.001
  };
\node[] at (axis cs: 13, 1.5e-3) {BER tham chiếu};
\end{axis}

\end{tikzpicture}

		\vspace{-1.5em}
		\caption{BER của OFDMA-MU-MIMO.} \label{fig:BER OFDMA-MU-MIMO}
	\end{subfigure}
	\begin{subfigure}[]{0.45\linewidth}
		% This file was created with tikzplotlib v0.10.1.
\begin{tikzpicture}[scale = 1]

\definecolor{darkturquoise0191191}{RGB}{0,191,191}
\definecolor{darkviolet1910191}{RGB}{191,0,191}
\definecolor{green01270}{RGB}{0,127,0}
\definecolor{lightgray204}{RGB}{204,204,204}
\definecolor{darktangerine}{RGB}{1.0, 0.66, 0.07}
% \scriptsize
  \begin{axis}[
  legend cell align={left},
  legend columns=2,
  legend style={
    fill opacity=0.8,
    draw opacity=1,
    text opacity=1,
    at={(0.01,0.15)},
    anchor=west,
    draw=lightgray204
  },
  log basis y={10},
  tick pos=left,
  xlabel={SNR [dB]},
  xmajorgrids,
  xmin=9.5, xmax=20.5,
  xminorgrids,
  xtick style={color=black},
  ylabel={BER},
  ymajorgrids,
  % ymin=9.28649163113646e-06, ymax=0.572645816836016,
  ymin=8.28649163113646e-11, ymax=0.572645816836016,
  yminorgrids,
  ymode=log,
  ytick style={color=black},
  % ytick={1e-09,1e-08,1e-07,1e-06,1e-05,0.0001,0.001,0.01,0.1,1,10},
  % yticklabels={
  %   \(\displaystyle {10^{-9}}\),
  %   \(\displaystyle {10^{-8}}\),
  %   \(\displaystyle {10^{-7}}\),
  %   \(\displaystyle {10^{-6}}\),
  %   \(\displaystyle {10^{-5}}\),
  %   \(\displaystyle {10^{-4}}\),
  %   \(\displaystyle {10^{-3}}\),
  %   \(\displaystyle {10^{-2}}\),
  %   \(\displaystyle {10^{-1}}\),
  %   \(\displaystyle {10^{0}}\),
  %   \(\displaystyle {10^{1}}\)
  % }
  ]
  %==========STA = 8
  \addplot [semithick, dashed, blue, mark=o, mark size=3, mark options={solid}]
  table {%
  10 0.105542086693548
  12 0.03317578125
  14 0.00980342741935484
  16 0.00169695060483871  
  18 0.000301108870967742 
  20 0.00008758064516129
  };
  \addlegendentry{\scriptsize STA=8, NRx=4 }
  \addplot [semithick, blue, mark=o, mark size=3, mark options={solid}]
  table {%
  10 0.00754063760080645
  12 0.000564415322580645
  14 0.0000329851310483871
  16 3.15020161290323e-07
  18 0
  20 0
  };
  \addlegendentry{\scriptsize STA=8, NRx=8 }
  %0.00330323840725806 18
 
  %========STA = 12
  \addplot [semithick, dashed, darkviolet1910191, mark=x, mark size=3, mark options=  {solid}]
  table {%
  10 0.219595619119624
  12 0.109079238071237
  14 0.0406001554099462
  16 0.0118438760080645
  18 0.00307266465053763
  20 0.000739961357526882
  };
  \addlegendentry{\scriptsize STA=12, NRx=4}
  \addplot [semithick, darkviolet1910191, mark=x, mark size=3, mark options={solid}]
  table {%
  10 0.0355962071572581
  12 0.00509545110887097
  14 0.000770770329301075
  16 5.35458669354839e-05
  18 4.70430107526882e-06
  20 2.52016129032258e-07
  };
  \addlegendentry{\scriptsize STA=12, NRx=8}

  %========STA = 16
  \addplot [semithick, dashed, darkturquoise0191191, mark=triangle, mark size=3,  mark options={solid}]
  table {%
  10 0.332619392641129
  12 0.215368353074597
  14 0.111528698336694
  16 0.0383162802419355
  18 0.0126934538810484
  20 0.00474168346774194
  };
  \addlegendentry{\scriptsize STA=16, NRx=4}

  \addplot [semithick, darkturquoise0191191, mark=triangle, mark size=3, mark   options={solid}]
  table {%
  10 0.0984311680947581
  12 0.0256324974798387
  14 0.0031804435483871
  16 0.000301631804435484
  18 0.0000225819052419355
  20 1.89012096774194e-06
  };
  \addlegendentry{\scriptsize STA=16, NRx=8}

  %========STA = 20
  \addplot [semithick, dashed, green01270, mark=square, mark size=3, mark options=  {solid}]
  table {%
  10 0.404963205645161
  12 0.324013810483871
  14 0.212407308467742
  16 0.116323966733871
  18 0.0588404737903226
  20 0.032302091733871
  };
  \addlegendentry{\scriptsize STA=20, NRx=4}
  %0.00169695060483871 18
  \addplot [semithick, green01270, mark=square, mark size=3, mark options={solid}]
  table {%
  10 0.196327394153226
  12 0.0706490675403226
  14 0.0148261340725806
  16 0.00330323840725806  
  18 0.000602318548387097
  20 0.000296421370967742
  };
  \addlegendentry{\scriptsize STA=20, NRx=8}

  %===========STA = 24
  \addplot [semithick, dashed, red, mark=diamond, mark size=3, mark options={solid}]
  table {%
  10 0.44402194640457
  12 0.391929918514785
  14 0.31088970094086
  16 0.21369783266129
  18 0.145292023689516
  20 0.115868048555108
  };
  \addlegendentry{\scriptsize STA=24, NRx=4}

  \addplot [semithick, red, mark=diamond, mark size=3, mark options={solid}]
  table {%
  10 0.281350806451613
  12 0.13714797547043
  14 0.039058110719086
  16 0.00865893817204301
  18 0.0025793220766129
  20 0.000844648857526882
  };
  \addlegendentry{\scriptsize STA=24, NRx=8}

  
  \addplot [ultra thick, loosely dashed, darktangerine, forget plot]
  table {%
  9.5 0.001
  20.5 0.001
  };
  \node[] at (axis cs: 13,3e-4) {BER tham chiếu};
  \end{axis}
\end{tikzpicture}


		\vspace{-0.25em}
		\caption{BER của OFDMA-IDMA.} \label{fig:BER OFDMA-IDMA}
	\end{subfigure}
	
	\caption{Hiệu suất BER trung bình giữa các kỹ thuật.}\label{fig:bersec4}
	\vspace{1.51em}
\end{figure}
% Although the reference BER is achieved, it is clear that OFDMA-IDMA still gives much better results (STA = 8, NRx = 8). In addition, when considering increasing the number of STAs to 20, the BER is still approximately $3\mathord{\times}10^{-4}$, meaning that even though the data rate is reduced by half, the number of STAs is increased 5 times compared to OFDMA.

% Figure \ref{fig:IT} shows the effect of the number of iterations on the BER performance of the proposed OFDMA-IDMA. The rising iteration number can reach a better BER. However, the latency must increase linearly with the number of iterations. Therefore, to ensure both latency and quality, the number of iterations chosen is 4, which gives the BER below the reference BER, and the latency will be below 10 $\mu s$ mentioned in \cite{UL-OFDM-IDMA}.

% In this context, four iterations are made. This selection yields a BER below the reference threshold and ensures that the latency remains within the specified threshold of 10 $\mu s$ discussed in \cite{UL-OFDM-IDMA}.